% - - - - - - - - - - - - - - - - - - - - - - - - - - - - - - - - - - - - - - -
% Mathematics
% - - - - - - - - - - - - - - - - - - - - - - - - - - - - - - - - - - - - - - -

\newcommand{\naturals} {\operatorname{\mathbb{N}}}
\newcommand{\integers} {\operatorname{\mathbb{Z}}}
\newcommand{\rationals}{\operatorname{\mathbb{Q}}}
\newcommand{\reals}    {\operatorname{\mathbb{R}}}
\newcommand{\complex}  {\operatorname{\mathbb{C}}}
\newcommand{\group}    {\operatorname{\mathbb{G}}}
\newcommand{\field}    {\operatorname{\mathbb{K}}}

\newcommand{\predicatename}[1]{\operatorname{\mathtt{#1}}}
\newcommand{\functionname}[1] {\operatorname{\mathsf{#1}}}
\newcommand{\constantname}[1] {\operatorname{\mathsf{#1}}}
\newcommand{\setname}[1]      {\operatorname{\mathbf{#1}}}

\renewcommand{\vert}{\,|\,}
\DeclareMathOperator{\divides}{\,\big|\,}
\let\oldtextvisiblespace\textvisiblespace%
\renewcommand{\textvisiblespace}{\text{\oldtextvisiblespace}}
\newcommand{\concat}{\cdot}

\newcommand{\orderLt} {\prec}
\newcommand{\orderLeq}{\preceq}
\newcommand{\orderGt} {\succ}
\newcommand{\orderGeq}{\succeq}

\newcommand{\restrict}[2]{\left.{#1}\right|_{#2}}
\newcommand{\domain}[1]  {\functionname{dom}({#1})}
\newcommand{\range}[1]   {\functionname{range}({#1})}

\DeclareMathOperator{\ld}{ld}
\DeclareMathOperator{\cross}{\times}
\DeclareMathOperator{\scalar}{\bullet}
\DeclareMathOperator*{\argmax}{arg\,max}
\DeclareMathOperator*{\argmin}{arg\,min}
\newcommand{\norm}[1]{||{#1}||}
\newcommand{\size}[1]{\functionname{size}({#1})}
\let\oldphi\phi
\renewcommand{\phi}{\varphi}
\newcommand{\definedas}{\coloneqq}
\newcommand{\asdefined}{\eqqcolon}

% set theory
\newcommand{\union}{\cup}
\newcommand{\intersection}{\cap}
\newcommand{\bigUnion}{\bigcup}
\newcommand{\bigIntersection}{\bigcap}

\newcommand{\powerset}[1]{%
    2^{#1}%
}
%\newcommand{\set}[1]{%
%    \left\{%
%        {#1}%
%    \right\}%
%}
%\newcommand{\setCondition}[2]{%
%    \left\{%
%        {#1}%
%    ~\middle|~%
%        {#2}%
%    \right\}%
%}
\newcommand{\tuple}[1]{%
    \left\langle%
        {#1}%
    \right\rangle%
}
\newcommand{\cardinality}[1]{%
    \left|%
        {#1}%
    \right|%
}
\newcommand{\ceil}[1]{%
    \left\lceil%
        {#1}%
    \right\rceil%
}
\newcommand{\floor}[1]{%
    \left\lfloor%
        {#1}%
    \right\rfloor%
}
\newcommand{\inverse}[1]{%
    \overline{#1}%
}

% - - - - - - - - - - - - - - - - - - - - - - - - - - - - - - - - - - - - - - -
% Logic
% - - - - - - - - - - - - - - - - - - - - - - - - - - - - - - - - - - - - - - -

% interpretations
\newcommand{\interA}{\mathcal{I}}
\newcommand{\interB}{\mathcal{J}}
\newcommand{\interC}{\mathcal{K}}
\newcommand{\varAss}{\mathcal{Z}}

% quantifiers
\newcommand{\holds}{\,.\,}
\newcommand{\st}{\holds}
\newcommand{\then}{\rightarrow}
\newcommand{\quantifier}{\mathord{\reflectbox{$\textsf{Q}$}}}

% entailment
% \models (already defined)
\newcommand{\entails}{\Vdash}

% provability
\newcommand{\proves}{\vdash}
\newcommand{\provesN}[1]{\vdash_{#1}}

% true and false
% \top (already defined)
\newcommand{\bottom}{\bot}
\newcommand{\ltrue} {\constantname{true}}
\newcommand{\lfalse}{\constantname{false}}

% logical connectives
% \land, \lor, \lnot (already defined)
\newcommand{\limplication}{\rightarrow}
\newcommand{\lequivalence}{\leftrightarrow}
% \to (already defined)
\newcommand{\from}{\leftarrow}
\newcommand{\lequal}{\approx}
\newcommand{\lnequal}{\not\approx}
\newcommand{\bigLand}{\bigwedge}
\newcommand{\bigLor}{\bigVee}
\newcommand{\lequiv}{\equiv}

% homomorphism
\newcommand{\isomorphic}{\cong}
\newcommand{\homomorphic}{\preceq}
\newcommand{\homom}{h}

% term sets
\newcommand{\constants} {\setname{Consts}}
\newcommand{\variables} {\setname{Vars}}
\newcommand{\functions} {\setname{Funcs}}
\newcommand{\nulls}     {\setname{Nulls}}
\newcommand{\terms}     {\setname{Terms}}
% predicate names
\newcommand{\predicates}{\setname{Preds}}

% functions
\newcommand{\arityName} {\functionname{arity}}
\newcommand{\arity}[1]  {\arityName({#1})}
\newcommand{\sigName}   {\functionname{sig}}
\newcommand{\sig}[1]    {\sigName({#1})}
\newcommand{\predsOf}[1]{\functionname{preds}({#1})}
\newcommand{\termsOf}[1]{\functionname{terms}({#1})}
\newcommand{\varsOf}[1] {\functionname{vars}({#1})}
\newcommand{\consOf}[1] {\functionname{consts}({#1})}

% substitutions
\newcommand{\subA}{\sigma}
\newcommand{\subB}{\theta}
\newcommand{\subC}{\tau}
\newcommand{\comp}{\circ}

% formulas
\newcommand{\forA}{\phi}
\newcommand{\forB}{\psi}
\newcommand{\forC}{\xi}

% - - - - - - - - - - - - - - - - - - - - - - - - - - - - - - - - - - - - - - -
% Description logic
% - - - - - - - - - - - - - - - - - - - - - - - - - - - - - - - - - - - - - - -

%\newcommand{\dlSubsumed}{\sqsubseteq}
%\newcommand{\dlEquiv}{\equiv}
%\newcommand{\dlAnd}{\sqcap}
%\newcommand{\dlOr}{\sqcup}
%\newcommand{\dlNot}{\neg}
%\newcommand{\dlUniv}[2]{\forall{#1} . {#2}}
%\newcommand{\dlExis}[2]{\exists{#1} . {#2}}
%\newcommand{\dlConAss}[2]{{#1} : {#2}}
%\newcommand{\dlRolAss}[2]{\left({#1}\right) : {#2}}
%
%\newcommand{\ALC}{\mathcal{ALC}}
%\newcommand{\dlABox}{\mathcal{A}}
%\newcommand{\dlTBox}{\mathcal{T}}
%\newcommand{\dlKB}{\mathcal{K}}
%
%\newcommand{\dlConA}{A}
%\newcommand{\dlConB}{B}
%\newcommand{\dlConC}{C}
%\newcommand{\dlRolA}{R}
%\newcommand{\dlRolB}{S}
%\newcommand{\dlRolC}{T}
%\newcommand{\dlIndA}{a}
%\newcommand{\dlIndB}{b}
%\newcommand{\dlIndC}{c}

% - - - - - - - - - - - - - - - - - - - - - - - - - - - - - - - - - - - - - - -
% Complexity theory
% - - - - - - - - - - - - - - - - - - - - - - - - - - - - - - - - - - - - - - -

\newcommand{\encoding}[1]{%
    \functionname{enc}\left(%
    {#1}
    \right)%
}
\newcommand{\bigO}[1]{\operatorname{\mathcal{O}}({#1})}

% asymptotically bounded
\newcommand{\DTime}[1] {\operatorname{\textsc{DTime}}({#1})}
\newcommand{\NTime}[1] {\operatorname{\textsc{NTime}}({#1})}
\newcommand{\DSpace}[1]{\operatorname{\textsc{DSpace}}({#1})}
\newcommand{\NSpace}[1]{\operatorname{\textsc{NSpace}}({#1})}
% logarithmic hierarchy
\newcommand{\ACZero}{\operatorname{\textsc{AC}}^0}
\newcommand{\LSpace}{\operatorname{\textsc{L}}}
\newcommand{\coL}   {\operatorname{\textsc{coL}}}
\newcommand{\NL}    {\operatorname{\textsc{NL}}}
\newcommand{\coNL}  {\operatorname{\textsc{coNL}}}
\newcommand{\NCn}[1]{\operatorname{\textsc{NC}}^{#1}}
\newcommand{\ACn}[1]{\operatorname{\textsc{AC}}^{#1}}
% polynomial time
\newcommand{\PTime}{\operatorname{\textsc{PTime}}}
\newcommand{\coP}  {\operatorname{\textsc{coP}}}
\newcommand{\NP}   {\operatorname{\textsc{NP}}}
\newcommand{\coNP} {\operatorname{\textsc{coNP}}}
% polynomial hierarchy
\newcommand{\hyperNP}[1]  {\operatorname{\mathrm{\Sigma}}^{\operatorname{p}}_{#1}}
\newcommand{\hyperCoNP}[1]{\operatorname{\mathrm{\Pi}}^{\operatorname{p}}_{#1}}
\newcommand{\hyperP}[1]   {\operatorname{\mathrm{\Delta}}^{\operatorname{p}}_{#1}}
% polynomial space
\newcommand{\PSpace}   {\operatorname{\textsc{PSpace}}}
\newcommand{\coPSpace} {\operatorname{\textsc{coPSpace}}}
\newcommand{\NPSpace}  {\operatorname{\textsc{NPSpace}}}
\newcommand{\coNPSpace}{\operatorname{\textsc{coNPSpace}}}
% exponential hierarchy
\newcommand{\ExpTime}      {\operatorname{\textsc{ExpTime}}}
\newcommand{\ExpSpace}     {\operatorname{\textsc{ExpSpace}}}
\newcommand{\NExpTime}     {\operatorname{\textsc{NExpTime}}}
\newcommand{\NExpSpace}    {\operatorname{\textsc{NExpSpace}}}
\newcommand{\nExpTime}[1]  {{#1}\ExpTime}
\newcommand{\nExpSpace}[1] {{#1}\ExpSpace}
\newcommand{\nNExpTime}[1] {{#1}\NExpTime}
\newcommand{\nNExpSpace}[1]{{#1}\NExpSpace}
% decidability border
\newcommand{\Elementary}{\operatorname{\textsc{Elemantary}}}
\newcommand{\Recursive} {\operatorname{\textsc{R}}}
\newcommand{\RE}        {\operatorname{\textsc{RE}}}
\newcommand{\coRE}      {\operatorname{\textsc{coRE}}}
% arithmetical hierarchy
\newcommand{\hyperSemidecidable}[1]  {\operatorname{\mathrm{\Sigma}}^0_{#1}}
\newcommand{\hyperCoSemidecidable}[1]{\operatorname{\mathrm{\Pi}}^0_{#1}}
\newcommand{\hyperDecidable}[1]      {\operatorname{\mathrm{\Delta}}^0_{#1}}

% - - - - - - - - - - - - - - - - - - - - - - - - - - - - - - - - - - - - - - -
% Paper-specific
% - - - - - - - - - - - - - - - - - - - - - - - - - - - - - - - - - - - - - - -

%\newcommand{\simScribe}{\functionname{scribe}}
%\newcommand{\simUnit}{\functionname{unit}}
%\newcommand{\simIden}{\functionname{iden}}
%\newcommand{\simComp}{\functionname{comp}}
%\newcommand{\simPair}{\functionname{pair}}
%\newcommand{\simTake}{\functionname{take}}
%\newcommand{\simDrop}{\functionname{drop}}
%\newcommand{\simInjL}{\functionname{injl}}
%\newcommand{\simInjR}{\functionname{injr}}
%\newcommand{\simCase}{\functionname{case}}
%\newcommand{\simWitness}{\functionname{witness}}
%\newcommand{\simFail}{\functionname{fail}}
%\newcommand{\simAssertL}{\functionname{assertl}}
%\newcommand{\simAssertR}{\functionname{assertr}}

\chapter{Efficient Circuits}

% execution on bit machine
% optimisations
% Minimize bandwidth and storage requirements by allowing sharing of expressions and pruning of unused expressions.

\section{Merkle Root}

\subsection{Identity Merkle Root}

The identity Merkle root (IMR) uniquely identifies a program at redemption time.
It is the counterpart of the CMR and is defined just as the latter,
with one important exception:
The IMR depends on the bitstring contained in witness combinators.
%
\begin{align*}
    \text{CMR}(\text{witness}\ a) &\definedas \text{tag}(\text{witness}) \\
    \text{IMR}(\text{witness}\ a) &\definedas \text{SHA256}_\text{Block}(\text{tag}(\text{witness}), a || \text{len}(a) || 0\ldots0)
\end{align*}
%
The CMR of a witness combinator is simply its tag,
independent of the bitstring.
Meanwhile,
the IMR of a witness combinator is computed as SHA256 of bistring $a$ with the tag as initial value.
\emph{(This computation is a simplification, because we deal with bitstrings.
In reality, Simplicity works in terms of Simplicity values. See Chapter~\ref{ch:verifiable} for more details.)}
As a consequence,
changing the witness changes the IMR.

\subsection{Sharing}

% compress program as much as possible
% there are no two parts that compute the same (roughly speaking)
% an attacker cannot change the program size while keeping the CMR the same (more precise)
% Merkle root of combinator identifies its computation
% witness combinators with different bitstrings compute different functions, so we must use IMR
% maximal sharing: there are no two combinators in the program that have the same IMR, input and output type
